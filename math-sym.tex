% fast font types
% \newcommand{\r}[1]{\mathrm{#1}}
% \newcommand{\c}[1]{\mathcal{#1}}


% define
\newcommand{\define}{\coloneqq}
% define sign from the right
\newcommand{\definedby}{\eqqcolon}
% function
\newcommand{\func}[3]{#1\colon#2\to#3}


% brackets
% curly brackets
\newcommand{\curlb}[1]{\left\{ #1 \right\}}
% box brackets
\newcommand{\boxb}[1]{\left[ #1 \right]}
% parentheses/curved brackets
\newcommand{\curvb}[1]{\left( #1 \right)}
% angle brackets
\newcommand{\angleb}[1]{\left\langle #1 \right\rangle}
% floor brackets
\newcommand{\floorb}[1]{\left\lfloor #1 \right\rfloor}
% ceil brackets
\newcommand{\ceilb}[1]{\left\lceil #1 \right\rceil}


% symbols for sets
% create sets
\newcommand{\set}[2][]{ \curlb{#2 \ifx&#1&\else \enspace\middle\vert\enspace #1 \fi} }
% standard sets
\newcommand{\SR}{\mathds{R}} % real numbers
\newcommand{\SC}{\mathds{C}} % complex numbers
\newcommand{\SN}{\mathds{N}} % natural numbers
\newcommand{\SZ}{\mathds{Z}} % integral numbers
\newcommand{\SQ}{\mathds{Q}} % rational numbers
\newcommand{\SFP}{\mathds{P}} % polynom functions
\newcommand{\SFC}{\mathrm{C}} % complex valued functions (continous or differentiable)
\newcommand{\SFL}{\mathcal{L}} % space of integrable functions
\newcommand{\SFLL}{\mathrm{L}} % space of integrable function classes
% set of linear maps
\newcommand{\LM}{L}
% hilbert space
\newcommand{\SH}{\mathcal{H}}
% set of matrices
\newcommand{\SM}{\mathrm{M}}
% set of invertible
\newcommand{\SGL}{\mathrm{Gl}}
% group of orthogonal matrices
\newcommand{\SO}{\mathrm{O}}
% special group of orthogonal matrices
\newcommand{\SSO}{\mathrm{SO}}
% group of unitary matrices
\newcommand{\SU}{\mathrm{U}}
% hauptraum/generalized eigenspace
\newcommand{\hau}{\mathrm{Hau}}


% elements
% identity
\DeclareMathOperator{\id}{id}
% identity matrix
\newcommand{\idmat}{\mathrm{I}}
% normal distribution
\newcommand{\FN}{\mathcal{N}}


% operators
% inverse
\newcommand{\inv}[1]{ {#1}^{-1} }
% magnitude/absolute value
\newcommand{\abs}[1]{\left\vert #1 \right\vert}
% norm
\newcommand{\norm}[1]{\left\| #1 \right\|}
% power of set
\DeclareMathOperator{\setpow}{\mathcal{P}}
% real part
\DeclareMathOperator{\real}{Re}
% imaginary part
\DeclareMathOperator{\imag}{Im}
% complex conjugate
\newcommand{\conj}[1]{ \overline{#1} }
% diagonal matrix
\DeclareMathOperator{\diag}{diag}
% trace of matrix
\DeclareMathOperator{\tr}{tr}
% kernel of function
% \DeclareMathOperator{\ker}{ker}
% image of function
\DeclareMathOperator{\im}{im}
% annihilator
\DeclareMathOperator{\ann}{ann}
% transponent matrix
\newcommand{\transp}[1]{ {#1}^\m{T} }
% spectrum of matrix
\DeclareMathOperator{\spec}{\sigma}
% rank of matrix
\DeclareMathOperator{\rank}{rank}
% signum of permutation or number
\DeclareMathOperator{\sign}{sign}
% expectation
\DeclareMathOperator{\expect}{\mathbb{E}}
% variance
\DeclareMathOperator{\var}{var}
% fourier transform
\newcommand{\fourier}{\mathcal{F}}
% derivative
\DeclareMathOperator{\Deriv}{D}
\newcommand{\deriv}[1]{ {#1}^{\prime} }
\newcommand{\dderiv}[1]{ {#1}^{\prime\prime} }
\newcommand{\ddderiv}[1]{ {#1}^{\prime\prime\prime} }
\newcommand{\nderiv}[2][]{ \ifx&#1& \deriv{#2} \else {#2}^{(#1)} \fi }
\DeclareMathOperator{\pderiv}{\partial}
% infinitesimal difference
\newcommand{\diff}{\mathrm{d}}
% integral
\newcommand{\integral}[4]{\int_{#1}^{#2} #3\ \diff #4}
\newcommand{\Integral}[4]{\int\limits_{#1}^{#2} #3\ \diff #4}
\newcommand{\iintegral}[2]{\int #1\ \diff #2} % indefinite integral
% scalar product
\newcommand{\dotp}[1]{\angleb{#1}}
% cross product sign
\newcommand{\cross}{\times}
% sign for direct sum
\newcommand{\dsum}{\oplus}
% linear span
\newcommand{\lspan}[1]{\angleb{#1}}
% dual space
\newcommand{\dual}[1]{ {#1}^* }
\newcommand{\ddual}[1]{ {#1}^{**} }

% converges arrow
\newcommand{\conv}[1][]{\xrightarrow[]{#1}}


% append unit
\newcommand{\unit}[1]{\, \mathrm{#1}}