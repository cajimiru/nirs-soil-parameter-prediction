\section{Simulation}
\label{sec:simulation}
	
	% it would be nice to have a figure, plotting the true SPSE
	% as an hline and for each simulation type a box plot (or
	% even cooler a violin plot).
	% if we use the following setup:
	% 1 at full resolution and full data set,
	% 3 at reduced resolution,
	% 3 at reduced data set
	% we have 7 boxplots/violins, which we can easily
	% plot in one figure of the size of fig 1.
	
	% The text that follows is a bogus text, capturing the style of the report here
	
	\subsection{Complete Simulations}
	\label{ssec:comp-sim}
	
		% report the results on the $\m{SPSE} - \widehat{\m{SPSE}}$
		Applying SANN on the 1000 simulations with the full design matrix showed that the $\widehat{\m{SPSE}}$ values underestimate the true $\m{SPSE}$ as displayed in figure \ref{fig:spse}.
		The mean $\widehat{\m{SPSE}}$ lies about $\mathcal{X}$ below its true value.
		It shall be noted though that the true $\m{SPSE}$ lies within one standard deviation of the estimate.
	
	% subsecton complete simulations
	
	\subsection{Resolution}
	\label{ssec:resolution}
	
		% report the results on the $\m{SPSE} - \widehat{\m{SPSE}} | m$
		Reducing the resolution at which we measure the lightwaves ameliorates the situation somewhat.
		Estimates lie much closer to the true $\m{SPSE}$, albeit the risk of obtaining outliers increases dramatically.
		The best result are yielded at a resolution of every $\frac{1}{2}$ of the original one.
	
	% subsection resolution
	
	\subsection{Sample Size}
	\label{ssec:samp-size}
	
		% report the results on the $\m{SPSE} - \widehat{\m{SPSE}} | l$
		% report the results on the $\m{SPSE} - \widehat{\m{SPSE}} | m$
		Reducing the data set at resolution $\frac{1}{3}$ shows even more dramatic effects on the variance of the estimates around the true $\m{SPSE}$.
		Compared to the full data set estimates at the same resolution, we notice a considerably wider interval around the mean.
		This is to be expected as lesser data increases the risk of variance.
	
	% subsection sample size

% section simulation