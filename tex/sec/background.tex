\section{Background}
\label{sec:background}
	
	\subsection{Soil Parameters}
	\label{ssec:soil-parameters}
	
		Let $A$ be any substance in a given soil sample dissolved in a solution of volume $V\in(0,\infty)$ and let $n_A\in(0,\infty)$ be the amount of $A$ in the sample.
		Then the molar concentration $c_A$ of $A$ is given by
		\[
			c_A \define \frac{n_A}{V}
		\]
		Now let $c_0$ be the molar concentration of this whole sample and $n_0$ the amount of the whole sample.
		We define the amount-of-substance fraction (ASF) $p_A$ of $A$ as 
		\[
			p^{(A)} \define 100\%\cdot\frac{c_A}{c_0} = 100\% \cdot\frac{n_A}{n_0}
		\]

		In this report, we concentrate on three important soil parameters that are given by existing NIRS-measurements. \cite{agelet:10a, don:a}
		The first two parameters $p^\m{(SOC)}$ and $p^\m{(N)}$ are the ASFs relating to soil organic carbon ($\m{SOC}$) and nitrogen ($\m{N}$) in a given soil sample.
		$\m{SOC}$ refers to the carbon in the sample that is bound in an organic compound.
		The third parameter is the $\m{pH}$-value that specifies the acidity of an aqueous solution.
		It links to the concentration of hydronium ions $c_\m{H_3O^+}$.
		\[
			\m{pH} \define -\lg c_\m{H_3O^+} = -\frac{\ln c_\m{H_3O^+}}{\ln 10}
		\]
	
	% subsection soil-parameters

	\subsection{Near Infrared Spectroscopy}
	\label{ssec:nirs}
	
		NIRS uses electromagnetic waves, \cite[246]{agelet:10a}
		 famously known as light, with wavelengths ranging from $780\unit{nm}$ to $3000\unit{nm}$, the so called near infrared spectrum.
		An emitted light wave with wavelength $\lambda \in (0,\infty)$ interacts with a soil sample in three ways:
		It can be reflected, absorbed or transmitted.
		For most soil samples measuring the transmittance of light waves is not sensible as the thickness of these samples varies.
		Since the absorptance cannot be determined directly, reflectance is used \cite[247-8]{agelet:10a}.
		

		Reflection can be split in specular and diffuse reflection.
		NIRS uses the latter as it penetrates the sample the most.
		As a consequence, diffuse reflected light is hemispherically scattered and contains information about the soil sample composition.
		For a more detailed view on this topic please refer to Agelet \citeyear[248-251]{agelet:10a}.

		The reflectance 
		\[
			\func{\varrho}{(0,\infty)}{(0,\infty)},\qquad \varrho(\lambda) \define \frac{P_\m{r}(\lambda)}{P_0}
		\]
		of a surface depending on wavelength $\lambda$ of a light wave is given by the amount of radiation power $P_\m{r}(\lambda)$ that is reflected from the surface divided by the initial power $P_0$ of the light wave.	
		In our case, this function $\varrho$ is defined as the near infrared spectrum of the soil sample.

		Absorptance itself originates from the existence of vibrational modes in molecules.
		A photon with a wavelength $\lambda \in (0,\infty)$ can only be absorbed if the appropriate frequency $f$ exactly matches a multiple of the transition energy of the bond or group that vibrates.
		This is why the spectra of soil samples are formed of overtones and combination bands \cite[247]{agelet:10a}.

		Due to the similarity of diffuse reflected and transmitted light we can use Beer-Lambert's law as an approximation of the relation of the attenuation of light and the properties of samples \cite[247-8]{agelet:10a}.
		Let $n\in\SN$ be the count of different substances in a sample and $c_i$ be the molar concentration of the $i$th substance for $i\in\SN,i\leq n$.
		Is again $\lambda \in (0,\infty)$ the wavelength of the used light then there exist certain coefficients $\varepsilon_i(\lambda)$ for all $i\in\SN,i\leq n$ such that

		\[
			-\ln \varrho(\lambda) = \sum_{i=1}^{n} \varepsilon_i(\lambda) c_i
		\]

	% subsection nirs

% section fundamentals