\section{Introduction}
\label{sec:introduction}
	
	Obtaining information on the composition of soil samples, such as the concentration of soil organic carbon ($\m{SOC}$), and other properties like the $\m{pH}$-value is important in answering a great variety of soil-related questions.	
	The direct measurement of soil composition is costly and error-prone \cite[1-3]{mclaughlin:99a}.
	Therefore methods that determine soil properties fast and cheaply play a fundamental and vital role in fields with a strong interest in soil analysis \cite{ludwig:01a}.

	Developed in the 1960s, Near Infrared Spectroscopy (NIRS) gained traction in soil science during the 1990s as sufficient computing power became increasingly available.
	NIRS provides a cheap alternative to other methods for the analysis of soil composition \cite[247]{agelet:10a}.

	Following a short summary of the chemo-physical background of NIRS, we will present some of the peculiarities of its application.
	We then proceed to describe in detail the methods combined to calibrate an NIRS system on a training set employing Mallow's $C_\m{p}$.
	To assess the robustness of the method, we investigate the algorithm aided by simulations and variations of two hyperparameters, resolution and training set size.
	We conclude after a short discussion of the implementation in \textsf{R} \cite{r-foundation:16a} and our results with an outline of additional investigations that seems promising for better results.

% section introduction