\section{Introduction}
\label{sec:introduction}
	
	Through soil analyses or soil testing one can get certain chemical and physical information about the used soil like the concentration of soil organic carbon (SOC) or the $\m{pH}$-value.
	With these Measurements, known as soil parameters, it is possible to assist in solving soil-related problems such as optimizing plant growth.
	% [Quelle: https://www.ndsu.edu/soils/services/soil_testing_lab/why_soil_test/]
	
	However the direct measurement of soil parameters is very costly and error-prone.
	Therefore methods for fast and cheap determination of these parameters play a fundamental and vital role in particular fields like agriculture, geochemistry and ecology.
	% [Quelle: https://en.wikipedia.org/wiki/Soil_test]

	Albeit developed in the 1960s, Near Infrared Spectroscopy (NIRS) only gained traction during the 1990s as sufficient computing power became increasingly available.
	NIRS provides a cheap alternative to other methods for the analysis of soil composition %such as%
	.

	After a short summary of the physical background of NIRS, we will lay out some of the difficulties in applying NIRS to real world problems.
	The rest of the paper is dedicated to a methodological framework to address some of those obstacles.
	In particular we propose a computationally efficient way to identify %near
	optimal parameters on a training sample with known data to be used for further analysis of NIRS on soil samples.

% section introduction