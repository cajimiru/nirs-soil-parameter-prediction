\section{Introduction}
\label{sec:introduction}
	
	Getting information on the composition of soil samples, such as the concentration of soil organic carbon ($\m{SOC}$), or other parameters like the $\m{pH}$-value is important in solving soil-related problems such as optimising plant growth \cite[1-3]{mclaughlin:99a}.	
	However the direct measurement of soil parameters is very costly and error-prone.
	Therefore methods for fast and cheap determination of these parameters play a fundamental and vital role in particular fields like agriculture, geochemistry and ecology \cite{ludwig:01a}.

	Albeit developed in the 1960s, Near Infrared Spectroscopy (NIRS) only gained traction during the 1990s as sufficient computing power became increasingly available.
	NIRS provides a cheap alternative to other methods for the analysis of soil composition \cite[247]{agelet:10a}.

	After a short summary of the physical background of NIRS, we will lay out some of the difficulties in applying NIRS to real world problems.
	The rest of the paper is dedicated to a methodological framework to address some of those obstacles.
	In particular, we implement and assess a computationally efficient way to calibrate a predictive model on a training set using Mallow's $C_\m{p}$ criterion.

% section introduction